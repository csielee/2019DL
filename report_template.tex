% --------------------------------------------------------------
% This is all preamble stuff that you don't have to worry about.
% Head down to where it says "Start here"
% --------------------------------------------------------------
 
\documentclass[12pt]{article}

\usepackage[margin=1in]{geometry} 
\usepackage{amsmath,amsthm,amssymb}
\usepackage[margin=1in]{geometry} 
\usepackage{amsmath,amsthm,amssymb}
%\usepackage[spanish]{babel} %Castellanización
\usepackage[T1]{fontenc} %escribe lo del teclado
\usepackage{inputenc} %Reconoce algunos símbolos
\usepackage{lmodern} %optimiza algunas fuentes
\usepackage{graphicx}
\graphicspath{ {images/} }
\usepackage{hyperref} % Uso de links

% To Display Chinese words
\usepackage{xeCJK}
% To Display Code
% \usepackage{listings}
\usepackage{minted}
% To Display csv file
\usepackage{csvsimple}
\usepackage{longtable}
\usepackage{booktabs}
 
\newcommand{\N}{\mathbb{N}}
\newcommand{\Z}{\mathbb{Z}}
 
\newenvironment{theorem}[2][Theorem]{\begin{trivlist}
\item[\hskip \labelsep {\bfseries #1}\hskip \labelsep {\bfseries #2.}]}{\end{trivlist}}
\newenvironment{lemma}[2][Lemma]{\begin{trivlist}
\item[\hskip \labelsep {\bfseries #1}\hskip \labelsep {\bfseries #2.}]}{\end{trivlist}}
\newenvironment{exercise}[2][Exercise]{\begin{trivlist}
\item[\hskip \labelsep {\bfseries #1}\hskip \labelsep {\bfseries #2.}]}{\end{trivlist}}
\newenvironment{problem}[2][Problem]{\begin{trivlist}
\item[\hskip \labelsep {\bfseries #1}\hskip \labelsep {\bfseries #2.}]}{\end{trivlist}}
\newenvironment{question}[2][Question]{\begin{trivlist}
\item[\hskip \labelsep {\bfseries #1}\hskip \labelsep {\bfseries #2.}]}{\end{trivlist}}
\newenvironment{corollary}[2][Corollary]{\begin{trivlist}
\item[\hskip \labelsep {\bfseries #1}\hskip \labelsep {\bfseries #2.}]}{\end{trivlist}}

\newenvironment{solution}{\begin{proof}[Solution]}{\end{proof}}

% write csv content in latex 
\begin{filecontents*}{grade.csv}
name,givenname,matriculation,gender,grade
Maier,Hans,12345,m,1.0
Huber,Anna,23456,f,2.3
Weisbaeck,Werner,34567,m,5.0
\end{filecontents*}
 
\begin{document}
 
% --------------------------------------------------------------
%                         Start here
% --------------------------------------------------------------
 
\title{2019 Deep Learning and Practice \\ Lab x -- xxx}
\author{0756110 李東霖}

\maketitle
\section{Introduction}

\begin{figure}[H]
\centering
% \includegraphics[width=\linewidth]{path/to/image} 
\caption{figure description}
\end{figure}

\section{Experiment setup}

% display code
\begin{minted}[frame=lines, breaklines]{python}
import numpy as np

np.arange(10)

iamlongnamevariablewithlongvalue = "iamverylongsoneedtobreaklineintoarticlewowowowowowwowowow"
\end{minted}

% display inline
\verb|I am inline code|

% display equation
\begin{equation}
\label{trick}
z = z^* * \exp(logvar/2) + mean
\end{equation} 

% use reference
I use equation reference ( \ref{trick} )

\section{Experimental results}

% display csv content
% \csvautotabular{path/to/csv.csv}

% display content in latex
\ \par
This table
\par
\csvautotabular{grade.csv}
\par \ \par

This book table with caption
\par
\begin{table}[H]
    \centering
    \csvautobooktabular{grade.csv}
    \caption{Hi i am caption}
    \label{tab:my_label}
\end{table}


\section{Discussion}

% --------------------------------------------------------------
%     You don't have to mess with anything below this line.
% --------------------------------------------------------------
 
\end{document}